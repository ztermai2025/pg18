\documentclass[lang=cn,newtx,10pt,scheme=chinese]{elegantbook}

\title{PostgreSQL DBA就业之路}
\subtitle{国人剖析 PostgreSQL 内核技术的原创之作}

\author{小布}
% \institute{Elegant\LaTeX{} Program}
% \date{2022/12/31}
% \version{4.5}
% \bioinfo{自定义}{信息}

\extrainfo{吾生也有涯,而知也无涯}

\setcounter{tocdepth}{3}

\logo{logo-blue.png}
\cover{cover.jpg}

% 本文档命令
\usepackage{array}
\newcommand{\ccr}[1]{\makecell{{\color{#1}\rule{1cm}{1cm}}}}

% 修改标题页的橙色带
\definecolor{customcolor}{RGB}{32,178,170}
\colorlet{coverlinecolor}{customcolor}
\usepackage{cprotect}

\addbibresource[location=local]{reference.bib} % 参考文献,不要删除

\begin{document}

\maketitle
\frontmatter

\tableofcontents

\mainmatter

\chapter{体系架构概述}
\section{创建实验环境}
\section{Linux前导知识}
\section{体系架构概述}

\chapter{数据文件}
\section{PostgreSQL源代码的基础知识}
\section{数据文件}
\section{性能测试工具pgbench的使用}

\chapter{理解WAL}
\section{WAL的基础知识}
\section{检查点}
\section{崩溃恢复}
\section{WAL文件的管理}

\chapter{备份和恢复}
\section{物理备份}
\section{增量物理备份}
\section{数据库的恢复}
\section{逻辑备份和恢复}

\chapter{流复制}
\section{快速搭建流复制}
\section{主库和备库的通讯过程}
\section{流复制的监控}
\section{主备库之间的切换}
\section{从备库上执行备份}

\chapter{堆表和B树索引}
\section{堆表}
\section{B树索引}

\chapter{多版本并发控制}
\section{事务}
\section{多版本并发控制}

\chapter{表和索引的清理(VACUUM)}
\section{一个VACUUM实验}

\chapter{逻辑复制}
\section{快速搭建逻辑复制}
\section{逻辑复制的体系架构}
\section{从备库进行逻辑复制}

\chapter{SQL性能优化}

\chapter{数据库升级}

\end{document}
